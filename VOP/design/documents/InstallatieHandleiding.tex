\documentclass[a4paper,11pt]{article}

\usepackage[dutch]{babel}
\usepackage{url}
\usepackage[latin1]{inputenc}
\usepackage{fullpage}
\usepackage{xspace}
\usepackage{graphicx}
\usepackage{listings}
\usepackage{hyperref}


\begin{document}

\pagenumbering{roman}

\title{Installatiehandleiding\\Vakoverschrijdend Project\\Groep 1}
\author{Harm Delva\\ Mathias De Brouwer\\ Maxime Fern\'andez Alonso\\ Jens Spitaels\\ Casper Van Gheluwe\\ Steven Van Maldeghem\\ Titouan Vervack}
\date{}
\maketitle

\pagenumbering{arabic}

\section*{Dependencies}
\begin{itemize}
	\item Gradle - gebruikt om de code te compileren, tests uit te voeren en de server te starten
	\item Java 8
	\item PostgresSQL
\end{itemize}

\section*{Installatie}
Er zijn twee manieren om te installeren; een manuele methode en een automatische methode waarvoor wat extra dependencies nodig zijn. We raden u aan de automatische methode te gebruiken. Ze worden echter beide besproken.
	\subsection*{Manuele installatie}
	\begin{enumerate}
		\item Maak in het PostgreSQL DBMS een nieuwe databank aan
		\item Voer de 7 DDL files in de directory \texttt{ddl} uit in alfabetische volgorde. Deze zorgen voor de gewenste databankstructuur.
		\item Voer een \texttt{INSERT}-query uit om de `root'-administrator aan te maken. Dit kan bijvoorbeeld als volgt:
			\begin{lstlisting}[frame=single, language=SQL, breaklines=true]
INSERT INTO person
(first_name, last_name, email, password, is_admin, is_analyst)
VALUES
('first', 'last', 'me@deloitte.com', 'paswoord', true, false);
			\end{lstlisting}
			Hierbij moet \texttt{paswoord} wel een geldige \hyperlink{http://bcrypt.sourceforge.net/}{bcrypt}-ge\"encrypteerde string zijn. Eventueel kan u dit laten doen door het Linux-commando \texttt{bcrypt}.
		\item Kopieer het bestand \texttt{application\-context\-example.xml} naar \texttt{application\-context.xml} en pas daarin de nodige databankconnectieparameters aan.
		\item Volg de instructies om de testomgeving op te zetten. Dit proces wordt uitgelegd in het bestand ``\href{run:TestHandleiding.pdf}{TestHandleiding.pdf}". Voor het starten van de server zullen immers ook de tests uitgevoerd worden.
		\item Start nu de server met het commando \texttt{./gradlew clean build run}
		\item De server begint te luisteren op \texttt{localhost}, poort \texttt{9000}. De poort kan naar wens aangepast worden in het bestand \texttt{src/main/resources/application.properties}.
	\end{enumerate}
	\subsection*{Automatische installatie}
	\begin{enumerate}
		\item Installeer de dependencies voor het \texttt{setup\_backend.py} script door het bash-script\\ \texttt{install\_dependencies.sh} uit te voeren.
		\item Maak in het PostgreSQL DBMS een nieuwe databank aan
		\item Voer volgend commando uit: ``\texttt{python setup\_backend.py setup\_db add\_admin context}''. Het script zal u vragen om de gewenste parameters en dan automatisch de databanktabellen instellen, een adminaccount installeren en de applicatiecontext juist initialiseren. Eventueel kan u ``\texttt{python setup\_backend.py --help}'' uitvoeren voor meer informatie.
		\item Volg de instructies om de testomgeving op te zetten. Dit proces wordt uitgelegd in het bestand ``\href{run:TestHandleiding.pdf}{TestHandleiding.pdf}". Voor het starten van de server zullen immers ook de tests uitgevoerd worden.
		\item Start nu de server met het commando \texttt{./gradlew clean build run}
		\item De server begint te luisteren op \texttt{localhost}, poort \texttt{9000}. De poort kan naar wens aangepast worden in het bestand \texttt{src/main/resources/application.properties}.
	\end{enumerate}
\end{document}