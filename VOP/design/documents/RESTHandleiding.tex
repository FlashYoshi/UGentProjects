\documentclass[a4paper,11pt]{article}

\usepackage[dutch]{babel}
\usepackage{url}
\usepackage[latin1]{inputenc}
\usepackage{fullpage}
\usepackage{xspace}
\usepackage{graphicx}

\begin{document}

\pagenumbering{roman}

\title{API Handleiding\\Vakoverschrijdend Project\\Groep 1}
\author{Harm Delva\\ Mathias De Brouwer\\ Maxime Fern\'andez Alonso\\ Jens Spitaels\\ Casper Van Gheluwe\\ Steven Van Maldeghem\\ Titouan Vervack}
\date{}
\maketitle

\pagenumbering{arabic}

\section*{Swagger}
Voor het ontwerp en controle van de door onze applicatie aangeboden REST interface, die in samenspraak met de andere groepen werd vastgelegd, maken we gebruik van de \textbf{Swagger} specificaties.
Deze leggen een formaat vast voor het documenteren van web APIs, waardoor het mogelijk is om er automatisch documentatie voor te genereren. Swagger heeft ook een tool die het mogelijk maakt om 
eenvoudig de responses op requests te controleren.

\section*{Gebruik}
U kan gebruik maken van \textbf{Swagger UI}. Deze kan u vinden op de URL\\
\noindent ``\href{http://vopro1.ugent.be/data/restapi/}''. Er verschijnt een lijst met tags, die elk verschillende operaties categorizeren.
Wanneer u op ��n van de tags klikt, laat de interface u toe om per operatie (bijvoorbeeld \texttt{POST /project}) de nodige parameters in te geven en te testen of het verwachte antwoord verschijnt. 
\end{document}