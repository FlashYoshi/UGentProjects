\documentclass[a4paper,11pt]{article}

\usepackage[dutch]{babel}
\usepackage{url}
\usepackage[latin1]{inputenc}
\usepackage{fullpage}
\usepackage{xspace}
\usepackage{graphicx}

\begin{document}

\pagenumbering{roman}

\title{Specificatiedocument\\Vakoverschrijdend Project\\Groep 1}
\author{Harm Delva\\ Mathias De Brouwer\\ Maxime Fern\'andez Alonso\\ Jens Spitaels\\ Casper Van Gheluwe\\ Steven Van Maldeghem\\ Titouan Vervack}
\date{}
\maketitle

\pagenumbering{arabic}

\section*{Specificatie}
Een geregistreerde gebruiker kan een nieuw project aanmaken, de naam van een bestaand project aanpassen of een bestaand project verwijderen. Naast projecten zijn er ook 3 soorten entiteiten: usecases, actoren en concepten. Entiteiten worden gekenmerkt door een naam en een projectid, dit wil dus zeggen dat ze allemaal een naam hebben die uniek is voor het project waartoe ze behoren.\\

\noindent Entiteiten kunnen net zoals projecten aangemaakt, aangepast of verwijderd worden. Actoren en concepten worden beide gebruikt in usecases, maar dezelfde actoren en concepten kunnen in meerdere usecases voorkomen. 

\end{document}