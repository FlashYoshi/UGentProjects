\documentclass[a4paper,11pt]{article}

\usepackage[dutch]{babel}
\usepackage{url}
\usepackage[latin1]{inputenc}
\usepackage{fullpage}
\usepackage{xspace}
\usepackage{graphicx}

\begin{document}

\pagenumbering{roman}

\title{Testhandleiding\\Vakoverschrijdend Project\\Groep 1}
\author{Harm Delva\\ Mathias De Brouwer\\ Maxime Fern\'andez Alonso\\ Jens Spitaels\\ Casper Van Gheluwe\\ Steven Van Maldeghem\\ Titouan Vervack}
\date{}
\maketitle

\pagenumbering{arabic}

\section*{Dependencies}
\begin{itemize}
	\item Gradle - gebruikt om de code en tests te compileren
	\item Java 8
	\item PostgresSQL - gebruikt als testdatabank
	\begin{itemize}
		\item U kan zelf de databankconnectieparameters instellen. Dit kan op twee manieren:
		\begin{itemize}
			\item \textbf{Installatiescript:} Voer \texttt{python setup\_backend.py test\_context} uit en geef de gevraagde parameters in.
			\item \textbf{Manueel:} Kopieer het bestand \texttt{spring\_test\_example.xml} naar \texttt{spring\_test.xml} en pas daarin de nodige parameters aan.
		\end{itemize}
		\item Alle tests worden volledig onafhankelijk van elkaar uitgevoerd; de tester moet zelf geen tabellen aanmaken of verwijderen. De databank wordt enkel aangesproken in de tests van de \texttt{Data Access Objects} zelf.
	\end{itemize}
\end{itemize}

\section*{Uitvoeren}
Na het installeren van \textbf{Gradle} en de \textbf{PostgreSQL} database kan u de tests uitvoeren door volgende commando's uit te voeren in de rootdirectory van het project:
\begin{itemize}
	\item op Unix-systemen: \texttt{gradle test}
	\item op Windows-systemen: \texttt{gradlew.bat test}
\end{itemize}
\noindent Gradle zal automatisch alle andere libraries installeren, de code compileren en daarna de testen uitvoeren. Indien alle tests slagen, zal er \texttt{BUILD SUCCESS} verschijnen.
\end{document}